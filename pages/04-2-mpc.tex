\subsection{MPC}
\begin{itemize}
    \item ``online (repeated) open loop computations $\corresponds$ feedback''
    \item repeated open-loop finite-horizon OC policy
(implemented in a receding horizon fashion)
\end{itemize}

\begin{optExample}{{1967, Marcus, Lee: Foundation of OC}}
    ``One technique for obtaining a feedback controller/synthesis from knowledge of open-loop controllers is to measure the current process control state and then compute \textbf{very rapidly} for the open loop control.\\
    
    ``The first portion of this [open-loop] function is used during a short time interval, after which a new measurement of the state is made and a a new open loop control function is computed for this measurement.\\
    
    ``Then the procedure is repeated.''
\end{optExample}

\paragraph{Model Predictive Control Scheme} ~\\
\begin{figure}[H]
    \centering
    \tikzsetnextfilename{mpc-scheme-timeline}
    \begin{tikzpicture}[
            every node/.append style={inner sep=0mm}
            ]
        \def\h{1cm}
        \node (origin) {} ;
        \node (k) [right=2.5*\h of origin] {};
        \node (kd) [right=1.9*\h of k] {};
        
        \begin{scope}[note]
            \node [below=1em of k.center] {$t_i$};
            \node [below=1em of kd.center] {$t_{i+1} = t_i + \delta$};
        \end{scope}
        
        \begin{scope}[arrow]
            \draw (origin) -- (kd) -- ++(1.5*\h,0) node (end) {};
            \node [right=3mm of end] {$t$};
        \end{scope}
        
        \begin{scope}[draw]
            \draw (k) -- ++(0,-2mm) -- ++(0,4mm);
            \draw (kd) -- ++(0,-2mm) -- ++(0,4mm);
        \end{scope}
    \end{tikzpicture}
\end{figure}

\begin{variables}
    T_p    & prediction/planning horizon\\
    \delta    & sampling interval
\end{variables}

\textbf{Typical scheme}:
\begin{enumerate}
    \item Measure the state $x(t_i)$
            and initialise it to be the current sate
    \item Predict the solution
            \begin{align*}
                \tund{u}{MPC}\paren{.,x\paren{t_i}}%\\
                &= \argmin \intdt{t_i}{t_i + T_p}{f_0 \paren{t,x,u}}\\
                &\qquad + \phi \paren{t_i + T_p, x\paren{t_i + T_p}}\\
                %
                \txt{s.t.} \quad
                \dot{x} &= f \paren{t,u,x}\\
                x_0 &= x(t_i) \quad    \txt{IC}\\
                x &\in \setX{}\\
                u &\in \setU{}
            \end{align*}%
    \item Implement $u\paren{t,x\paren{t_i}}$ within the sampling interval, $t \in \squareb{t_i, t_{i+1}}$
        \lat{} `closed loop'
    \item Set value at end of sampling period $t_{i+1}$
    to be the new initial value of the next iteration\\
    \lat{} step 1 $i \leftarrow i + 1$
\end{enumerate}

\begin{figure}[H]
    %\centering
    \def\svgwidth{0.9\columnwidth}
    {\scriptsize \input{./img/mpc-scheme-iter1.pdf_tex}}
\end{figure}

\begin{figure}[H]
    %\centering
    \def\svgwidth{\columnwidth}
    {\scriptsize \input{./img/mpc-scheme-iter2.pdf_tex}}
%
\begin{align*}
    &\vdots\\
    \Rightarrow \txt{predicted solution} &\neq \txt{closed loop solution}\\
    \tund{u}{MPC}\paren{t, x\paren{t_i}} &
\end{align*}
\end{figure}%

\paragraph{Discussion / Challenges} ~\\
MPC $\subseteq$ (online) optimisation-based control
\begin{figure}[H]
    \centering
    \tikzsetnextfilename{mpc-opt-in-loop}
    \begin{tikzpicture}[every node/.append style={font=\footnotesize}]
        \def\h{1cm}
        \node [block] (plant) {plant};
        \node [block,below=0.7*\h of plant.south] (opt) {opt\\algo.};
        
        \begin{scope}[arrow]
            \draw (plant.east) -- ++(0.7*\h,0) |- (opt)
                            node [pos=0.5,right] {$y\paren{t_k,x_k}$};
            \draw (opt.west) -- ++(-0.7*\h,0)
                        |- (plant) node [pos=0.5,left] {$u_k$};
        \end{scope}
    \end{tikzpicture}
\end{figure}

Fast and reliable (feasible) algorithms needed%
\footnote{fast: opt algo has to be faster than the plant; reliable: algo not allowed to crash}!\\

\paragraph{Sequential/recursive feasibility} ~\\
``if solution at $t_i$ is feasible, guarantee that there exists
a solution at $t_{i+1}$''

\begin{gather*}
    \min \intdt{t_i}{t_i+T_p}{\dots} + \phi \quad \txt{is feasible}\\
    \Downarrow\\
    \min \intdt{t_{i+1}}{t_{i+1}+T_p}{\dots} + \phi \quad \txt{feasible?}
\end{gather*}

\paragraph{Stability of MPC} ~\\
\textbf{Problem:} predicted trajectory $\neq$ closed-loop trajectory

\begin{figure}[H]
    \centering
    \def\svgwidth{0.5\columnwidth}
    { \footnotesize
    \input{./img/mpc-stability.pdf_tex}
    }
\end{figure}

\classDate{Mo.\\10/12/18\\exercise}
\textbf{Solution:} ``change/modify objective function, constraints, prediction horizons ...''

%%%%%%%% Original notes from Fr 7/12/18
% \begin{enumerate}
%     \item \textbf{Zero-terminal constraint} $x(t_i + T_P) = 0$
%         \begin{figure}[H]
%             \centering
%             \def\svgwidth{0.65\columnwidth}
%             \input{./img/mpc-stability-ztc.pdf_tex}
%         \end{figure}
%     \item \textbf{Terminal set/terminal cost}
%         \begin{align*}
%             x(t_i + T_P) \in \setX{}_f
%         \end{align*}
%         \begin{figure}[H]
%             \centering
%             \def\svgwidth{0.5\columnwidth}
%             \input{./img/mpc-stability-term-set.pdf_tex}
%         \end{figure}
%     \item Choose \textbf{$T_p$ sufficiently large}\\
%             ``Disadvantage: no. of decision variables increases, more computation power needed''
%     \item (Terminal set-free, terminal cost-free,\\
%             ... etc.)
% \end{enumerate}
%%%%%%%%%%%%

\classDate{Fr.\\14/12/18}

% \begin{itemize}
%     \item Terminal set based conditions/MPC schemes
%     \item Terminal set free approaches
% \end{itemize}~ 

\begin{itemize}
    \item Terminal set based conditions/MPC schemes
            \begin{enumerate}
                \item Terminal set constraint\\
                        $x(t_i + T_p) \in \setX{f}$
                        \begin{figure}[H]
                            \centering
                            \def\svgwidth{0.5\columnwidth}
                            { \footnotesize
                            \input{./img/mpc-stability-term-set.pdf_tex}
                            }
                        \end{figure}
                \item Zero terminal state constraint\\
                        Special case of the above, where 0 is the desired equilibrium point, $x(t_i+T_p) = 0$
                        \begin{figure}[H]
                            \centering
                            \def\svgwidth{0.5\columnwidth}
                            { \footnotesize
                            \input{./img/mpc-stability-ztc.pdf_tex}
                            }
                        \end{figure}
            \end{enumerate}
    \item Terminal set free approaches\\
            e.g. choose \textbf{$T_p$ sufficiently large}\\
            Disadvantage: no. of decision variables increases, more computation power needed, $T_p$ difficult to quantise
\end{itemize}

\begin{theorem}{A stability theorem}{}
    Consider 
    \begin{align*}
        V \paren{  x(t_i) }
        = \min  & \intdt{t_i}{t_i+T_p}{ f_0 \paren{x(t), u(t)} } \\
                    &+ \phi \paren{ x\paren{ t_i + T_p } }\\
        \st \quad   & \dot{x}(t) = f \paren{ x(t), u(t) }, \\
                    &\qquad t \in [ t_i, t_i + T_p ]\\
                    & u(t) \in \setU{}\\
                    & x(t) \in \setX{}\\
                    & \txt{IC } x(t_i)\\
                    & x\paren{ t_i + T_p } \in \setX{f}
    \end{align*}
    
    which is solved at each sampling time $t_i$, $t_{i+1} = t_i + \delta$
    and implemented in a receding horizon fashion.\\
     
\end{theorem}

\begin{theorem}{(cont.)}{}
    Suppose
    \begin{itemize}
        \item $f(0,0) = 0$\\
                $f_0$, $\phi$ are positive definite\\
                $\phi \in C^1$
        \item $V$ well-defined in a set $\setX{0} \subseteq \R^n$ of initial conditions, where the problem is feasible.\\
                $V \in C^1$ in (interior) of $\setX{0}$\\
                $V$ positive definite
        \item $0 \in \setX{f} \subseteq \setX{0} \subseteq \setX{}$
        \item There exists a `local feedback controller' $k(x)$ such that
                \begin{enumerate}
                \renewcommand{\labelenumi}{\bfseries A\arabic{enumi}.}
                    \item $k(x) \in \setU{}, x \in \setX{}$
                    \item $\dot{x} = f \paren{ x, k(x) }$ renders $\setX{f}$ invariant, i.e.
                        \begin{align*}
                            x(0) &\in \setX{f}\\
                            x(t) &\in \setX{f}, t>0
                        \end{align*}
                    \item
                        \begin{align*}
                            \dot{\phi}(x)   &= \nabla \phi^T (x) f \paren{ x, k(x) } \\
                                            &\leq - f_0 \paren{ x, k(x) } \qquad
                        \end{align*}
                \end{enumerate}
    \end{itemize}~
    
    $\Rightarrow$ Then the MPC scheme is recursively  feasible and converges $x(t_i) \lat{} 0$, $i \lat{} \infty$ \bluetext{and the MPC closed loop is asymptotically stable for $\delta \lat{} \infty$} 
\end{theorem}

\paragraph{Proof} ~\\
\begin{figure}[H]
    \centering
    \inkscape{mpc-feas-proof}{.8}
\end{figure}

\begin{enumerate}
    \renewcommand{\theenumi}{\alph{enumi}}
    \renewcommand{\labelenumi}{\theenumi )}
    \item Recursive feasibility
    \item Convergence
\end{enumerate}~

\paragraph{a) Recursive feasibility}
OCP is feasible for $x(t_i) = $ IC\\
$\Rightarrow$ OCP is feasible for  $x(t_{i+1}) = $ IC.

\begin{figure}[H]
    \centering
    \inkscape{mpc-sets}{.6}
\end{figure}

Candidate solution (suboptimal, feasible solution at $t_{i+1}$:
\begin{align*}
    \nu \paren{ t, x \paren{ t_{i+1} } } &= \left\lbrace
        \begin{array}{l>{t \in }l}
            u^*(t)                         & \squareb{ t_i + \delta, t_i + T_p }\\
            \tilde{u} = k \paren{ x(t) }   & \squareb{ t_i + T_p, t_{i+1} + T_p }
        \end{array} \right.
\end{align*}

We have:
\begin{align*}
    x \paren{ t_{i+1} + T_p } &\in \setX{f} \qquad &\because (A2)\\
    \nu \paren{ t_i, x \paren{ t_{i+1} } } &\in \setU{} &\because (A1)
\end{align*}~

$\Rightarrow$ recursive feasibility \\
(state constraints ok, since $\setX{f}$ is invariant and $\setX{f} \subseteq \setX{}$)

\paragraph{b) Convergence} $x(t_i) \lat{} 0$
\def\colRed{\color[rgb]{0.725,0.047,0.278}}
\def\colBlue{\color[rgb]{0.047,0.133,0.725}}
\def\colPurple{\color[rgb]{0.633, 0.064, 0.658}}
\begin{align*}
    V \paren{ x \paren{ t_{i+1} } }
    &\overtext{by def./opt.}{\leq} 
    {\colBlue
        V \paren{ x \paren{ t_{i+1}, \nu } }
    }
    \overset{?}{<}
    \colRed V \paren{ x \paren{ t_i } }\\
    %
    \colRed V \paren{ x\paren{ t_i } }
    &= \colRed \intdt{t_i}{t_i + T_p}{
        f_0 \paren{ x^*, u^* }
        }\\
        &
        \colRed \quad
        + \phi \paren{
            x^*\paren{ t_i + T_p }
            }\\
    \colBlue V \paren{ x\paren{t_{i+1}}, \nu } & \colBlue =
        \intdt{t_{i+1}}{t_i + T_p}{
        f_0(x^*,u^*)
        } \\
        & \colBlue \quad + \intdt{t_i + T_p}{t_{i+1}+T_p}{ f_0 (\tilde{x}, \tilde{u}) }\\
        & \colBlue \quad + \phi \paren{ \tilde{x} \paren{ t_{i+1} + T_p } }
\end{align*}

\begin{figure}[H]
    \centering
    \inkscape{mpc-convergence}{1}
\end{figure}

${\colBlue V(x(t_{i+1}, v) } <
{\colRed V(x(t_i)) }$ if\\
endpiece area $<$ cost difference
\begin{align*}
    {\colBlue \intdt{t_i + T_p}{t_{i+1}+T_p}{ f_0 (\tilde{x}, \tilde{u}) } }
    &<
    {\colRed \phi \paren{ x^* \paren{ t_{i} + T_p } } }\\
    & \quad -
    {\colBlue
        \phi \paren{ \tilde{x} \paren{ t_{i+1} + T_p } }
    }
\end{align*}

${\colBlue V \paren{ x(t_{i+1}), \nu } }
= {\colRed V( x(t_i) )} + $ Rest
{
\scriptsize
\begin{align*}
      &
      \colBlue{}
      V(x(t_{i+1}), \nu)\\
    = & 
    \colBlue{}
    {
    \colPurple{}
    \intdt{t_{i+1}}{t_i+T_p}{f_0(x^*,u^*)}
    }
    +
    \intdt{t_i+T_p}{t_{i+1}+T_p}{f_0(\tilde{x},\tilde{u})}
    + \phi \paren{ \tilde{x} \paren{ t_{i+1} + T_p } }\\
    & {
        \colRed +\intdt{t_i}{t_{i+1}}{f_0 (x^*,u^*)}
        }
        - \intdt{t_i}{t_{i+1}}{f_0 (x^*,u^*)}\\
    & {
        \colRed +
        \phi \paren{ x^*(t_i + T_p) }
        }
        - \phi \paren{ x^*(t_i + T_p) }
\end{align*}
}%
%
\begin{align*}
    { \colBlue V\paren{ x(t_{i+1}, \nu} }
    \leq
    \colRed V(x(t_i))
        & \colBlue + \phi \paren{ \tilde{x} \paren{ t_{i+1} + T_p } }\\
        & \colRed - \phi \paren{ x^*(t_i + T_p) }\\
        & \colBlue + \intdt{t_i+T_p}{t_{i+1}+T_p}{f_0(\tilde{x},\tilde{u})}
\end{align*}

If
${\colBlue \phi \paren{ \tilde{x}  \dots } }
{ \colRed - \phi \paren{ x^* \dots } }
{ \colBlue + \intdt{t_i+T_p}{t_{i+1}+T_p}{f_0 \dots } } \leq 0$ (A3)

\paragraph{Remark}
\begin{gather*}
    \begin{align*}
        \min \sumk{j=k}{k+N_p-1}
        f_0 \paren{ x(j), u(j) }
        + \phi \paren{ x \paren{ k + N_p } }
    \end{align*}\\
    \begin{align*}
        \st{} \quad
        x(j+1) &= f( x(j), u(j)),\\
        &\quad j = k, \dots, k + N_p - 1\\
        u(j) &\in \setU{}\\
        x(j) &\in \setX{}\\
        \txt{IC } &= x(k)\\
        &\quad \txt{current state of the plant}\\
        x(k+N_p) &\in \setX{}
    \end{align*}
\end{gather*}

\paragraph{Stability proof} see exercise\\
\begin{align}
    x_k \in \setX{f} &\Rightarrow x_{k+1} \in \setX{f}\\
        &\qquad x_{k+1} = f ( x_k, \overline{k}(x_k) ) \nonumber\\
    x_k \in \setX{f} &\Rightarrow \overline{k} (x_k) \in \setU{}\\
    \phi(x_{k+1}) - \phi(x_k) &\leq -f_0 (x_k, \overline{k}(x_k))
\end{align}

$\Rightarrow$ parametrised, time-varying NLP
\begin{align*}
    \min \quad  & f(u,p) \qquad p = x(k)\\
    \st \quad   & g(u,p) \leq 0
\end{align*}

\paragraph{Pros and cons of MPC}
\begin{itemize}
    \item easy to handle state and input constraints
    \item good model needed
    \item fast and reliable algorithm needed
\end{itemize}

\classDate{Mo.\\17/12/18\\exercise}
~\vspace{1em}\\
