\def\rad{1.1cm}
\def\gupw{10pt}
\def\guph{0.5cm}
\def\mshift{1.5cm}
\begin{tikzpicture}[
    w1/.style={shape=circle,minimum size=2*\rad,text width=2cm,
                inner sep=0pt,outer sep=0pt,
                draw,semithick,align=flush center,
                font=\footnotesize},
    ]
% Wagen W1
\node (w1) [w1]
    {\textbf{W1}\\Motor, Getriebe, Ritzel};
    
% Ground
\node [ground,minimum width=3cm,below=0cm of w1.south] (ground) {};
\draw (ground.north east) -- (ground.north west);

\node (gup) [ground,minimum height=\guph,minimum width=\gupw,
            above=\guph of w1.north,
            xshift=-0.5*\gupw] {};
\draw (gup.south east) -- (gup.north east);

% Traegheit
\draw [arrow] (gup.east) arc (90:45:\rad+\guph)
        node [above=10pt,pos=0.9] {$\varphi, \omega, \alpha$};

% Motormoment
\node at ($(w1) +(50:\mshift)$) (Mm) {};
\draw [force] (Mm) arc (50:-10:\mshift)
        node [right=5pt,pos=0.5] {$M_M1$};
\end{tikzpicture}